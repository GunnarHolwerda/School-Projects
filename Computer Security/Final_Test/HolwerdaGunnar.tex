\documentclass[12pt]{report}
\usepackage{graphicx}
\usepackage{listings}

% Document setup
\graphicspath{ {img/} }
\title{CSCI-476 Final Test}
\date{May 6, 2015}
\author{Gunnar Holwerda}
\renewcommand*\contentsname{Table of Contents}
\newcommand{\mychapter}[2]{
    \setcounter{chapter}{#1}
    \setcounter{section}{0}
    \chapter*{#2}
    \addcontentsline{toc}{chapter}{#2}
}

% Begin actual report

\begin{document}
\maketitle

\tableofcontents
\clearpage

\mychapter{1}{Executive Summary}
\section{Executive Summary}

\clearpage

% Begin Round 1

\mychapter{2}{Round 1}
\section{DiscoveredIn1655}
When starting out I was given the information that the members of mhk had been discussing RFC2100. This RFC mentions a few names, so I began using the names mentioned as subdomains of mthack.me and quickly found titan.mthack.me. I ran nmap on the host to see what ports were open.\\
	\begin{verbatim}
	$	nmap -sS -p1-65535 titan.mthack.me -v -T4
	\end{verbatim}
The nmap returned that port 22 and 23 were open. I attempted to ssh, but found that a public key was needed. Next I used telnet to connect to port 23 and was presented with my first flag ``DiscoveredIn1655''.\\
\includegraphics[scale=0.33, width=\linewidth]{DiscoveredIn1655.png}
\newline
\section{Th1sT1m3ItsAMoon}
In addition to titan.mthack.me, I was able to find the europa.mthack.me subdomain. After an nmap on europa I saw that port 7870 was open. There was no information about this port, so I used NetCat to connect to it, it returned ``SSH-2.0-OpenSSH\_6.6.1''. After seeing this I knew that I should use SSH to connect to europa.mthack.me on this port.
	\begin{verbatim}
	$	ssh europa.mthack.me -p 7870
	\end{verbatim}
After adding europa to my known\_hosts I was presented with my second flag ``Th1sT1m3ItsAMoon''.\\
\includegraphics[scale=0.33, width=\linewidth]{Th1sT1m3ItsAMoon.png}
\newline
\section{SocialContract}
Another subdomain that I found during my investigation was hobbes.mthack.me. When visiting the subdomain in a browser you are presented with a webpage that says ``hi''. I decided to try to run dirbuster to see if there were any files that were accessible on the server through the browser. After many attempts with the built in directory wordlists in Kali, I downloaded a new set and ran the ``Crazy'' version of the wordlist. This returned that /~root/ssh was accessible. There was a file in the directory leading me to the ``.ssh'' folder. I downloaded the private key from the server using:\\
	\begin{verbatim}
	$	wget hobbes.mthack.me/~root/.ssh/id_rsa
	\end{verbatim}

I then moved the key to my .ssh directory and ran:
	\begin{verbatim}
	$	ssh thomas@hobbes.mthack.me
	\end{verbatim}

There was a text file in the home directory, I opened it up, and found the final flag in Round 1 ``SocialContract''.\\
\includegraphics[scale=0.33, width=\linewidth]{SocialContract.png}
\clearpage

% Begin Round 2

\mychapter{3}{Round 2}
\section{TomcatIsAVulnerability}
The information for Round 2 led me to mercury.mthack.me. I used nmap on the host and discovered that port 8089 was open, but was unknown. After using netcat to try to tell what the service was, I opened up mercury.mthack.me:8089 in my browser. I was then presented with the Tomcat homepage. I tried to login as the manager with ``tomcat'', ``tomcat'' as my username and password and was successful in logging in.\\
After trying multiple exploits and payloads through metasploit, I eventually found a website that provided me with a payload that I could upload which gave me a shell through the browser. I uploaded that file and started navigating the filesystem looking for a file. After about 20 minutes of unsuccessul searches I decided to start listing everything ordered by time. I figured that the flag would be in a directory that had been modified most recently. This strategy led me to /opt/ where the flag.txt file was residing. I opened the file and was presented with the flag from round 2, ``TomcatIsAVulnerability''.\\
\includegraphics[scale=0.33, width=\linewidth]{TomcatIsAVulnerability.png}
\newline

% Begin Round 3

\mychapter{4}{Round 3}
\section{nextlevel}
Given the binary for round three, I first ran strings on the file using grep to try to find ``password'' or something along those lines. These attempts were unsuccessful, so I moved onto editing the binary using radare2. I was able to find the location of a ``jnz'' instruction right after asking for the number. I edited that instruction to be a ``jz'' instead and was presented with ``ciph3rfun.html''.\\

\includegraphics[scale=0.33, width=\linewidth]{ciph3rfunhtml.PNG}

I then visited www.mthack.me/ciph3rfun.html and was presented with some sort of enocded flag. It looked like ROT, so I went to a ROT decoder, entered the cipher text and was presented with the flag ``nextlevel''.\\
\includegraphics[scale=0.33, width=\linewidth]{nextlevel.PNG}
\newline

% Begin Round 4

\mychapter{5}{Round 4}
\section{FSInc3ption}

	\begin{verbatim}
	# hydra ssh://192.168.2.12 -l user -P /usr/share/wordlists/metasploit/namelist.txt -v -t 8
	\end{verbatim}
\includegraphics[scale=0.33, width=\linewidth]{FSInc3ption.PNG}
\newline
\section{deaddrop}
%
%	Text for deaddrop
%
\includegraphics[scale=0.33, width=\linewidth]{deaddrop.png}
\newline

\mychapter{6}{Round 5}
\section{haxtheplanet}
\includegraphics[scale=0.33, width=\linewidth]{haxtheplanet.png}


\mychapter{7}{Misc. Flags}
\section{mylittlepwnie}
%
%	Text for mylittlepwnie
%
\includegraphics[scale=0.33, width=\linewidth]{mylittlepwnie.png}
\clearpage

\mychapter{8}{Summary}

\clearpage

\mychapter{9}{Bibliography}

\end{document}